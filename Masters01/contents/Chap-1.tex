\chapter{Introduction}
\label{chp:intro}


%%%%%%%%%%%%%%%%%%%%%%%%%%%%%%%%%%%%%%%%%%%%%%%%%%%%%%%%%%%%%%%%%%%%%%%
\section{Background}

Unmanned Aerial Vehicles (UAVs) are a technology that have gained popularity in various applications recently\cite{CPP-Survey-2019}. Originally, UAVs required a ground pilot to manoeuvre them, but are becoming an increasingly automated technology. Applications where UAV automation has been used include, but are not limited to, structure inspections\cite{Guerrero2013}, smart farming\cite{Lottes2017}, \\disaster management\cite{Maza2011}, power line inspections\cite{Chang2017}, surveillance\cite{Basilico2015} and wildfire tracking\cite{Pham2017}.

Most of the applications mentioned were done using multi-rotor UAVs, quad-rotor vehicles in particular. It is important to note that the term UAVs also encompasses other aircraft types, like single rotor and fixed wing UAVs. Hybrids also exist that contain both rotary-wing and fixed-wing components\cite{CPP-Survey-2019}.

Using UAVs poses a considerable advantage in applications like the ones mentioned when compared to unmanned ground vehicles (UGVs). Their capacity to fly over landscapes and around three dimensional structures makes their potential functionality increase substantially. Relatively high altitude flying is a key reason why they are well suited to the application suggested in this paper, which is coverage path planning for search and rescue missions.

Coverage path planning (CPP) is a variant of the general motion planning problem. Originally, motion planning algorithms were predominantly used to find solutions for start-goal problems\cite{Choset2001}. This implies finding a sequence of actions to get an object from some starting state to some goal state. An example would be finding a sequence of movements to get a robotic arm from one orientation to another. In the context of path planning it means getting an agent, a UAV for example, from some starting position to some goal position in an environment\cite{Lynch2017}.

Coverage path planning is different from start-goal path planning in that it tries to determine a path for an agent to pass over all points in an environment\cite{Choset2001}. This type of coverage path planning is useful in applications like...
* Cpp

* multiple UAVs

The scope of this paper looks specifically at the application of UAVs in search and rescue operations. Manually piloted UAVs have already been integrated into search and rescue missions... 

Furthermore, there have been developments in the use of drones to perform automated search and rescue operations. One such example is a project by DroneSAR where they use DJI drones to perform search rescue tasks. Their implementation focuses on the automatic coverage of a manually designated area using a single drone. This implementation is the first of it's kind and has already showed promise in shortening the time it takes to locate victims. Included in their design is imaging and tracking technology to help locate victims in search and rescue operations.




Automated 

UAVs - manual operation vs automated flight
What is CPP
Talk about SAR


\section{Objectives}

The main objective of this thesis was to develop a coverage path planning algorithm that utilises multiple UAVs to search a known environment. This research is intended to be applicable to search and rescue operations.\linebreak
\cite{CPP-Survey-2019}


\section{Methodology}

\section{Scope of the Research}

\section{Thesis Structure}
