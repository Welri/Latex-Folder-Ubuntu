\chapter{Introduction}
\label{chp:intro}


%%%%%%%%%%%%%%%%%%%%%%%%%%%%%%%%%%%%%%%%%%%%%%%%%%%%%%%%%%%%%%%%%%%%%%%
\section{Background}

Unmanned Aerial Vehicles (UAVs) are a technology that have gained popularity in various applications recently\cite{CPP-Survey-2019}. Originally, UAVs required a ground pilot to manoeuvre them, but are becoming an increasingly automated technology. Applications where UAV automation has been used include, but are not limited to, structure inspections\cite{Guerrero2013}, smart farming\cite{Lottes2017}, disaster management\cite{Maza2011}, power line inspections\cite{Chang2017}, surveillance\cite{Basilico2015} and wildfire tracking\cite{Pham2017}.

Most of the research mentioned was done on the premise of using multi-rotor UAVs, quad-rotor vehicles in particular. It is important to note that the term UAVs also encompasses other aircraft types, like single rotor and fixed wing UAVs. Hybrids also exist that contain both rotary-wing and fixed-wing components\cite{CPP-Survey-2019}.

Using UAVs poses a considerable advantage in applications like the ones mentioned when compared to unmanned ground vehicles (UGVs). Their capacity to fly over landscapes and around three dimensional structures makes their potential functionality increase substantially. Relatively high altitude flying is a key reason why they are well suited to the application suggested in this paper, which is coverage path planning for search and rescue missions.

Coverage path planning (CPP) is a variant of the general motion planning problem. Originally, motion planning algorithms were predominantly used to find solutions for start-goal problems\cite{Choset2001}. This implies finding a sequence of actions to get an object from some starting state to some goal state. An example would be finding a sequence of movements to get a robotic arm from one orientation to another. In the context of path planning it means getting an agent, a UAV for example, from some starting position to some goal position in an environment\cite{Lynch2017}.

Coverage path planning is different from start-goal path planning in that it tries to determine a path for an agent to pass over all points in an environment\cite{Choset2001}. It can be used with ground vehicles, for example, to automate field machines for smart farming\cite{Hameed2014}. Further examples include vacuum cleaning robots, spray painting robots\cite{Atkar2005}, window cleaning robots\cite{Mir-Nasiri2018} and automated lawn mowers\cite{Arkin1999}. For underwater vehicles it can be used for the inspection of difficult to reach underwater structures\cite{Englot2012}. 

Furthermore, there have been developments in the use of UAVs to perform automated search and rescue operations. Perhaps the most notable example is a project by DroneSAR where they use DJI drones to perform search and rescue tasks. Their implementation includes a mobile app that allows the user to designate a search area manually\cite{DroneSAR01}. This is called off-line coverage, where the environment is known prior to the path planning step. The entire path is computed before being performed by the UAV. Online coverage allows the path to change or be computed during path execution. Often times the environment is not known a priori\cite{CPP-Survey-2019}. 

There is a downside regarding computation time in an off-line approach, but complete coverage is more achievable in an off-line approach. Online approaches are very useful for situations where the environment is constantly changing or there is very little known about the environment. Search and rescue operations often span large areas and UAVs fly above most ground obstacles. Therefore, it is realistic to assume the environment can be mapped prior to the search operation, especially since complete coverage is rather important in a search and rescue operation.\cite{CPP-Survey-2013}

DroneSAR uses one drone per search and rescue operation. 

\section{Objectives}

The main objective of this thesis was to develop a coverage path planning algorithm that utilises multiple UAVs to search a known environment. This research is intended to be applicable to search and rescue operations.\linebreak
\cite{CPP-Survey-2019}


\section{Methodology}


\section{Scope of the Research}


\section{Thesis Structure}
% Later