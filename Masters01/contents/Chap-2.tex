\chapter{Literature Review}
\label{chp:back}

% Give some kind of intro to CPP and the different approaches - Surveys
% zhang paper
% What is motion planning - Lavalle
As mentioned in section \ref{sec:intro_bg}, coverage path planning is a subset of the general motion planning problem. Therefore, to understand CPP one must first understand motion planning. This chapter will therefore begin with a brief overview of motion planning before moving on to a more detailed review of CPP literature. Beyond that, it presents an overview of literature regarding UAVs and their applications in Search and Rescue.
\section{Motion Planning}
Perhaps one of the most notable items of literature presented on motion planning is \cite{Lavalle2006}. This book refers to planning algorithms as a strategy that is used by one or more decision makers to move from some starting state to some goal state. They often refer to the user of  the algorithm as a \emph{decision maker}. Another popular term, which is taken from the field of artificial intelligence is the term \emph{agent}, which will be the  term used throughout this text.
\section{Coverage Path Planning}

%%%%%%%%%%%%%%%%%%%%%%%%%%%%%%%%%%%%%%%%%%%%%%%%%%%%%%%%%%%%%%%%%%%%%%%
% Mention which ones are grid-based etc. etc.
\section{Single Robot Coverage Path Planning}
\subsection{A-Star}
\subsection{STC}
\section{Multi-Robot Coverage Path Planning}

\subsection{Divide Areas Algorithm}
% Describe wwhat DARP is
% Explain why divide areas was used and specifically why DARP was chosen.
% Make reference to single-robot CPP for individual area searching
% Mention other forms of divide area algorithms?
% Discuss the shortcomings of the DARP algorithm and how some people have addressed them. That one paper used a method where they exchange cells
\subsection{MSTC/MFC}
% MSTC and MFC

\section{UAVs and Search and Rescue}
\subsection{Rotary-Wing UAVs}
\subsubsection{DroneSAR}
\subsection{Fixed-Wing UAVs}


