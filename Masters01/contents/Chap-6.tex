\chapter{Hardware Design with Software Implementation}
\label{chp:hwsw}


%%%%%%%%%%%%%%%%%%%%%%%%%%%%%%%%%%%%%%%%%%%%%%%%%%%%%%%%%%%%%%%%%%%%%%%
\section{Scope}

In granular or particle flow simulations with Discrete Element Method (DEM),
the mechanical behavior of a system of particles are simulated. The basic
building blocks of DEM are finite sized particles and walls. It is generally
classified into two basically different approaches.

The first is the ``hard sphere'', event-driven method
\citep[e.g.][]{Luding-1994, Luding-2004}, where particles are assumed to be
perfectly rigid and they follow an undisturbed motion until a collision
occurs. Due to the rigidity of the interaction, the collisions occur
instantaneously with accompanying momentum transfer. It is mainly used for
collisional, dissipative granular gases.

The second is the so-called ``soft particle'' molecular dynamics pioneered by
%\citet{Cundall-1979}, where the particles are allowed to overlap or penetrate
each other. Constrains on the physical space that a particle can occupy at a
specific time is included with contact or penalty forces related to the
amount of overlap and contact velocity between particles or between particles
and walls. The motion of the system is modelled by the integration of
Newton-Euler equations for motion of every individual particle.
